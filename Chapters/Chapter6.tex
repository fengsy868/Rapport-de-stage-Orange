% Chapter 6

\chapter{Conclusion} % Main chapter title

\label{Chapter6} % For referencing the chapter elsewhere, use \ref{Chapter6} 

\lhead{Chapter 6. \emph{Conclusion}} % This is for the header on each page - perhaps a shortened title

%----------------------------------------------------------------------------------------

The past six months of my internship have been very instructive for me. It's the first time that I participated in an industrial research project. This opportunity allows me to learn and develop myself in many areas. I gained a lot of experience, especially in scientific programming and parallel computing. A lot of activities are familiar with what I have learned at school. They enhanced my competence in HPC and inspired my interests in industrial research work.

On the other hand, I also encountered problems that I had never come across before. It was good to find out what my weaknesses are. This helped me to define which skills and knowledges I have to improve in the coming study time. 

One thing I appreciate a lot is the HPC coffee organized by group members. They invite researchers specialized in one particular area to share their opinions and ideas with other people. All participants exchange their study issues comfortably and freely. Even if I didn't understand all discussions, I did get a global picture for my unknown topics from these discussions. Besides, If I hadn't had the chance to participate the L'Ecuyer's presentation at CEA, I might never know that generating random numbers is a very interesting topic. Such experience makes me eager to learn more and explore more; it really broadened my horizons.

In addition, every day's coffee time offered me a great chance to get involved in a French life. Traveling, sports, technique or children, all daily life came into conversation. These enriched my sight in a "French" way and I found myself greatly improved on my french.

From this internship I also learned the importance of time management. Once I realized what I had to do, I organized my day and work so that I did not waste my hours. A Gantt diagram of my internship helps with the time management (see \aref{AppendixC}).

This six-month internship passed too quickly. I am grateful for meeting so many wonderful staff members. I will try my best to apply all I learned here into my future careers and sincerely hope that I have the chance to come back.